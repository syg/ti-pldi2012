\section{Related Work}
\label{sec:related-work}

There is an enormous literature on points-to analysis, JIT compilation, and
type inference. We only compare against a few here.

The most relevant work on type inference for JavaScript to the current work is
Logozzo and Venter's work on rapid atomic type analysis \cite{Logozzo10}.
Like ours, their analysis is also designed to be used online in the context of
JIT compilation and must be able to pay for itself. Unlike ours, their
analysis is purely static and much more sophisticated, utilizing a theory of
integers to better infer integral types vs floating point types. We eschew
sophistication in favor of simplicity and speed. Our evaluation shows that
even a much simpler static analysis, when coupled with dynamic checks, performs
very well ``in the wild''.
Our analysis is more
practical: we have improved handling of what Logozzo and Venter termed ``havoc''
statements, such as \code{eval}, which make static analysis results
imprecise. As Richards et al. argued in their surveys, real-world use of
\code{eval} is pervasive, between 50\% and 82\% for popular websites
\cite{Richards11, Richards10}.

Other works on type inference for JavaScript are more formal. The work of
Anderson et al. describes a structural object type system with subtyping over
an idealized subset of JavaScript \cite{Anderson05}. As the properties held
by JavaScript objects change dynamically,
the structural type of an object is a flow-sensitive property. Thiemann
and Jensen et al.'s typing frameworks approach this problem by using recency
types \cite{Thiemann05, Jensen09}. The work of Jensen et al. is in the context
of better tooling for JavaScript, and their experiments suggest that the
algorithm is not suitable for online use in a JIT compiler. Again, these
analyses do not perform well in the presence of statically uncomputable
builtin functions such as \code{eval}.

Performing static type inference on dynamic languages has been proposed at
least as early as Aiken and Murphy \cite{Aiken91}. More related in spirit to
the current work are the works of the the implementors of the Self language
\cite{Ungar87}. In implementing type inference for JavaScript, we faced many
challenges similar to what they faced decades earlier \cite{Ungar92,
  Agesen94}. Agesen outlines the design space for type inference algorithms
along the dimensions of efficiency and precision. We strived for an
algorithm that is both fast and efficient, at the expense of requiring
runtime checks when dealing with complex code.

Tracing JIT compilers \cite{GalVEE09, GalPLDI09}
have precise information about the types of expressions,
but solely using type feedback limits the optimizations that can
be performed.
Reaching peak performance requires static knowledge about
the possible types of heap values.

Agesen and H\"olzle compared the static approach of type inference with the
dynamic approach of type feedback and described the strengths and weaknesses
of both \cite{Agesen95}. Our system tries to achieve the best of both
worlds. The greatest difficulty in static type inference for polymorphic
dynamic languages, whether functional or object-oriented, is the need to
compute both data and control flow during type inference. We solve this by
using runtime information where static analyses do poorly,
e.g. determining the particular field of a polymorphic receiver or the
particular function bound to a variable. Our type barriers may be seen as a
type cast in the context of Glew and
Palsberg's work on method inlining \cite{Glew02}.

Framing the type inference problem as a flow problem is a well-known approach
\cite{Oxhoj92, Palsberg91}; practical examples include Self's inferencer
\cite{Agesen93TI}.
%Generalizing Hindley/Milner-style type inference to type
%set inclusion from type set equality has been approached at least as early as
%Mitchell's work on type inference and Thatte's work on partial types
%\cite{Mitchell84, Thatte88}.
Aiken and Wimmers presented general results on
type inference using subset constraints \cite{Aiken93}.
%, and Wand showed how to
%frame inference of principal types in function languages as a constraint
%problem \cite{Wand87}.
%Our work, formally, is much simpler than many such
%proposed solutions to solving type constraints by choice.
More recently, Rastogi et al. has described a gradual type inference algorithm
for ActionScript, an optionally-typed language related to JavaScript
\cite{Rastogi12}. They recognized that unlike static type systems, hybrid type
systems need not guarantee that every use of a variable be safe for every
definition of that variable. As such, their type inference algorithm is also
encoded as a flow problem. Though their type system itself is hybrid --- it
admits the \emph{dynamic} type --- their analysis is nevertheless static. As
it is designed to help evolve an untyped script to a typed one, they do not
add additional dynamic checks to increase precision.

% No
%unification is required to arrive a solution --- a solution always exists by
%simple propagation.

Other hybrid approaches to typing exist, such as Cartwright and Fagan's soft
typing and Taha and Siek's gradual typing \cite{Cartwright91, Siek07}. They
have been largely for the purposes of correctness and early error
detection, though soft typing has been successfully used to eliminate runtime checks \cite{Wright97}.
%Soft typing inserts run-time checks, and gradual typing focuses on
%the interaction of code with type annotation and code without
%annotation.
We say these approaches are at least partially \emph{prescriptive}, in that they
help enforce a typing discipline, while ours is entirely \emph{descriptive}, in that
we are inferring types only to help JIT compilation.


% Notes on other things to cite:
% We take efficiency of analyses to be used in JIT compilers as the ability to pay for themselves, reference chez scheme.
% Say somewhere and cite that pointer analysis scales well in the wild despite O(n^3).
% Future work: perhaps talk about pluggable static portions? parametrised. CFA2, Chambers's Fast Interprocedural Class Analysis paper

%%% Local Variables: 
%%% mode: latex
%%% TeX-master: "main"
%%% End: 
