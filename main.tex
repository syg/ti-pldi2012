\documentclass[preprint]{sigplanconf}

\usepackage[T1]{fontenc}
\usepackage[scaled=0.80]{beramono}
\usepackage{amsmath}
\usepackage{stmaryrd}
\usepackage{amsfonts}
\usepackage{amssymb}
\usepackage{amsthm}
\usepackage{listings}
\usepackage{url}
\usepackage{mathpartir}

%\newcommand{\code}[1]{{\scriptsize {\tt #1}}}
\newcommand{\code}[1]{\texttt{#1}}
\newcommand{\judge}{\vdash}
\newcommand{\rulename}[1]{{[\textsc{#1}]\quad}}

% Comment out lineno for final draft.
\usepackage{lineno}
\linenumbers

% JavaScript language definition.
\lstdefinelanguage{JavaScript}
{morekeywords={var,let,function,new,if,else,switch,while,catch,try,finally,throw,true,false,eval,with,undefined,case,default,do,break,for,in,continue,return},
  keywordstyle=\ttfamily\bfseries,
  sensitive=false,
  comment=[l]{//},
  morestring=[b]",
  morestring=[b]',
  mathescape=true}
\lstset{captionpos=t,xleftmargin=5pt,columns=flexible,basicstyle=\ttfamily,showstringspaces=false,escapechar=\#,language=JavaScript}

\title{Fast and Precise Hybrid Type Inference for JavaScript}
% Author info left blank for double blind submission.
\authorinfo{}{}{}

\begin{document}

\maketitle

\begin{abstract}
JavaScript performance is often bound by its dynamically-typed
nature. Compiler writers do not have access to static type information, making
generation of efficient, type-specialized machine code difficult. To avoid
incurring extra overhead on the programmer and to improve the performance of
already-deployed JavaScript programs at large, we seek to solve this problem
by inferring types. Existing type inference algorithms for JavaScript,
however, are often too computationally intensive and too
imprecise---especially in the case of JavaScript's extensible objects---to
enable optimizations. Both problems arise from performing purely static
analyses. In this paper we present a hybrid type inference algorithm for
JavaScript based on points-to analysis. Our algorithm is \emph{fast}, in that
it pays for itself in the optimizations it enables.  Our algorithm is also
\emph{precise}, by augmenting static analysis with run-time type barriers.

We also showcase an implementation for Mozilla Firefox's JavaScript engine,
demonstrating both performance gains and viability.
Through integration with the just-in-time (JIT) compiler in Firefox, we have improved
its performance on major benchmarks and JavaScript-heavy websites by up to 30\%.
This is tentatively scheduled to become the
default compilation mode in Firefox 9.
\end{abstract}

%\input{intro.tex}

\section{Example and Motivation}
\label{sec:example}

\begin{figure}
\begin{lstlisting}[xleftmargin=18pt]
function Box(v) {
  this.f = v;
}

function use(a) {
  var res = 0;
  for (var i = 0; i < 1000; i++) {
    var v = a[i].f;
    res = res + v;
  }
  return res;
}

function main() {
  var a = [];
  for (var i = 0; i < 1000; i++)
    a[i] = new Box(10);
  use(a);
}
\end{lstlisting}
\caption{Motivating Example}
\label{fig:motivating-example}
\end{figure}

Consider the example JavaScript program in Figure~\ref{fig:motivating-example}.
This program constructs an array of \code{Box} objects wrapping integer
values, then calls a \code{use} function which adds up the contents of all
those \code{Box} objects.
No types are specified for any of the variables or other values used
in this program, in keeping with JavaScript's dynamically typed nature.
However, most operations in this program interact with type information,
and knowledge of the involved types is needed for the code to
be efficiently compiled.

In particular, we are interested in the addition
\code{res + v} on line 9.
In JavaScript, it is legal to add any two values together, with the
operands coerced into strings or numbers if necessary. String
concatenation is performed for the former, and numeric addition for the latter.

Without static information about the types of \code{res} and \code{v},
a JIT compiler must emit code to handle all possible combinations of
operand types.
Moreover, every time values are copied around, the compiler must emit
code to keep track of the types of the involved values, using either
a separate type tag for the value or a specialized marshaling format.
This incurs a large runtime overhead on the generated code,
greatly increases the complexity of the compiler,
and makes effective implementation of important optimizations like
register allocation and loop invariant code motion much harder.

If we know the types of \code{res} and \code{v}, we can compile
code which performs an integer addition without type checks and the need
to track the types of \code{res} and \code{v}.
With static knowledge of all types involved in the program, the compiler can
in many cases generate code similar to that produced for a statically
typed language such as Java, with similar optimizations.

We can infer possible types for \code{res} and \code{v} statically
by reasoning about the effect the program's assignments
and operations have on values produced later.
This is illustrated below (for brevity, this reasoning does not consider
the possibility of functions \code{Box} and \code{use} being overwritten).

\begin{enumerate}
\item On line 17, \code{main} passes an integer when constructing \code{Box}
      objects. On line 2, \code{Box} assigns its parameter to the result's
      \code{f} property. Thus, \code{Box} objects can have an
      integer property \code{f}.
\item Also on line 17, \code{main} assigns a \code{Box} object to an element
      of \code{a}. On line 15, \code{a} is assigned an array literal,
      so the elements of that literal could be \code{Box} objects.
\item On line 18, \code{main} passes \code{a} to \code{use}, so \code{a}
      within \code{use} can refer to the array created line 15.
      When \code{use} accesses an element of \code{a} on line 8,
      per \#2 the result can be a \code{Box} object.
\item On line 8, property \code{f} of a value at \code{a[i]} is assigned to \code{v}.
      Per \#3 \code{a[i]} can be a \code{Box} object, and per \#1 the
      \code{f} property can be an integer. Thus, \code{v} can be an integer.
\item On line 6, \code{res} is assigned an integer. Since \code{v} can be
      an integer, \code{res + v} can be an integer.
      When that addition is assigned to \code{res} on line 9, the assigned
      type is consistent with the known possible types of \code{res}.
\end{enumerate}

This reasoning can be captured with inclusion constraints; we compute
sets of possible types for each value and model the flow of values between
these sets as subset relationships.
To be useful for compilation, we need to know not just \emph{some}
possible types for variables, but \emph{all} possible types.
In this sense, the static inference above is unsound, and does not account
for all possible behaviors of the program.
A few such behaviors are described below.

\begin{itemize}

\item The read of \code{a[i]} may access a {\it hole} in the array.
Out of bounds array accesses in JavaScript do not throw an exception,
but instead produce the \code{undefined} value if the array's prototype
does not have a matching property.
Such holes can also be in the middle of an array;
assigning to just \code{a[0]} and \code{a[2]} leaves a missing
value at \code{a[1]}.

\item Similarly, the read of \code{a[i].v} may be accessing a property
which is not actually held by \code{a[i]} or a prototype, and may produce the
\code{undefined} value.

\item The addition \code{res + v} may overflow.
JavaScript has a single number type which does not distinguish between
integers and doubles.
However, it is extremely important for performance that JavaScript compilers
distinguish the two and try to represent numbers as
integers wherever possible.
An addition of two integers may overflow and produce a number which can
only be represented as a double.

\end{itemize}

In some cases these behaviors can be proved statically to not occur,
but usually they cannot be ruled out.
If we try to capture these behaviors statically, many element or property
accesses will be marked as possibly undefined and many integer operations
will be marked as possibly overflowing.
The resulting type information will be too imprecise to be useful for
JIT compilation.

We address this problem by combining static inference about types
with targeted dynamic type updates.
Behaviors which are not accounted for statically must be caught dynamically,
which will trigger modification of the inferred types to reflect
those new behaviors.
If \code{a[i]} accesses a hole, the inferred types for the result must be
marked as possibly \code{undefined}.
If \code{res + v} overflows, the inferred types for the result must be
marked as possibly a double.

With or without analysis, the generated code needs to test for array holes
and integer overflow in order to correctly model the semantics of the language.
These and other dynamic type updates are {\it semantic triggers}:
they are placed on already existing, rarely taken execution paths
and incur a cost to update the inferred types only the first time that
execution path is taken for an operation.

The presence of these triggers illustrates the key invariant our analysis
preserves: inferred types must conservatively model all types for
variables and object properties which currently exist, but not those
which could exist in the future.
This has important implications:

\begin{itemize}

\item The program can be analyzed incrementally, as code starts to execute.
Code which has never executed does not need to be analyzed.
This is necessary for JavaScript due to dynamic code loading
and generation through
\code{eval()} and similar mechanisms. It is also important for reducing
analysis time on websites, which often load several megabytes of
code and only execute a fraction of it.

\item Assumptions about types made by the JIT compiler can be invalidated
at almost any time.
This affects the correctness of the JIT-compiled code, and the virtual machine
must be able to support recompiling or discarding that code at any time,
including (and especially) when that code is on the stack.
We discuss our handling of this in Section~\ref{sec:recompilation}.

\end{itemize}

This invariant is also critical to our handling of polymorphic code
within a program.
Consider the following extension to the example program:

\begin{lstlisting}[numbers=none]
function other() {
  var v = new Box("hello!");
}
\end{lstlisting}

Executing the \code{other} function will cause \code{Box} objects
to be created which hold strings,
illustrating the use of \code{Box} as a polymorphic structure.
Our analysis does not distinguish \code{Box} objects created in different
places, and the result of the \code{a[i].v} access in \code{use} will
be regarded as potentially producing a string.
Naively, solving the constraints produced by the analysis will mark
\code{a[i].v}, \code{v}, \code{res + v}, and \code{res} as all producing
either an integer or a string, even if \code{use}'s runtime behavior is actually monomorphic
and only works on \code{Box} objects containing integers.

This problem of imprecision leaking across the program is serious, as even
if a program is by and large monomorphic, the precision of analysis results on
it can easily be poisoned by a small amount of polymorphic code.

We deal with uses of polymorphic structures and functions using runtime checks.
At all element and property accesses, we keep track of both the set of
types which \emph{could} be observed for the access and the set of types
which \emph{has} been observed.
The former will be a superset of the latter, and if the two are different then
we insert a runtime check, a {\it type barrier}, to check for conformance
between whatever value is read out and the observed type set.
Mismatches lead to updates of the observed type set, and because our
invariant only requires inferred types to reflect observed types, it is
safe to only solve analysis constraints with respect to the observed types.

For the example program, a type barrier is required on the \code{a[i].f} access
on line 8, and nowhere else. The barrier will test that the value being read
is an integer. If a string shows up due to a call to \code{use}
outside of \code{main}, then the possible types of the \code{a[i].f} access
will be updated, and \code{res} and \code{v} will be marked as possibly
strings by resolving the analysis constraints.

Type barriers differ from the semantic triggers described earlier in that
they are not required by the semantics of the language and do not need to
be performed if our analysis is not being used.
These checks are cheap, though: if the set of observed types is monomorphic,
as is generally the case (Section~\ref{sec:access_polymorphism}),
only a single test is required by the barrier.
We are effectively betting that the barriers
pay for themselves by enabling generation of better code using more precise type information.
We have found this to be the case (Section~\ref{sec:without_barriers}).

Our contributions are:

\begin{itemize}

\item We describe the use of inclusion constraints to incompletely infer
possible types in JavaScript programs, and the runtime extension with semantic
triggers to generate sound type information.

\item We describe the use of type barriers to efficiently and precisely handle
polymorphic code, with a minimum of runtime overhead.

\item We showcase an implementation integrated with the JIT compiler used in
Firefox, and describe the issues encountered and fixed while preparing
the analysis for release.

\item We evaluate a variety of metrics showing the effectiveness of the analysis
and modified compiler on benchmarks and a selection of popular websites.

\end{itemize}

%%% Local Variables: 
%%% mode: latex
%%% TeX-master: "main"
%%% End: 


\section{Analysis}
\label{sec:analysis}

We present our analysis in two parts, the static ``may-have-type'' analysis
and the dynamic ``must-have-type'' analysis. The algorithm is based on
Andersen-style (inclusion based) pointer analysis \cite{AndersenPhD}. The
static analysis is
intentionally unsound with respect to the semantics of
JavaScript. It does not account for all possible behaviors of expressions and
statements and only generates constraints that model a ``may-have-type''
relation. All behaviors excluded by the type constraints must be detected at
runtime and their effects on types in the program dynamically recorded. The
analysis runs in the browser as functions are trying to execute: code is
analyzed function-at-a-time.

Inclusion based pointer analysis has a worst-case complexity of $O(n^3)$
and is very
well studied. It has shown---and we reaffirm this with our evaluation---to
perform and scale well despite its cubic worst-case complexity
\cite{Sridharan09}.

%% This paragraph and list doesn't seem to fit in the analysis section, this is more about performance. -ed
%
%Since the analysis runs in a browser as code is trying to execute, both time and space are at a premium and must be minimized. To be viable for deployment, our analysis must satisfy the following:
%
%\begin{itemize}
%
%\item Time and memory used must be linear in the size of the analyzed code.
%
%\item Querying results for an expression must take near-constant time, to avoid changing the performance characteristics of the compiler.
%
%% What does this mean? Interaction between "must" and "if only" is strange. -ed
%\item Results must be correct if only compiled code is analyzed.
%
%\end{itemize}

We describe constraint generation and checks required for a simplified core of
JavaScript expressions and statements, shown in Figure~\ref{fig:js-core}. We
let $f,x$ range over variables, $p$ range over property names, $i$ range over
integer literals, and $s$ range over string literals. The only control flow in
the core language is \code{if}, which tests for definedness. We avoid talking
about functions and function calls in our simplified core; the reader may
think of functions as objects with special domain and codomain properties.

The types over which we are trying to infer are also shown in
Figure~\ref{fig:js-core}. The types can be primitive or an object type
$o$.\footnote{In full JavaScript, we also have the primitive types \code{bool}
  and \code{null}.} The \code{int} type indicates a number expressible as a
signed 32-bit integer and is subsumed by \code{number} --- \code{int} is
added to all type sets containing \code{number}. Finally, we have sets of
types which the static analysis computes.

\newcommand{\barrier}{\supseteq_\mathcal{B}}

\begin{figure}
\begin{align*}
v & ::= \code{undefined}\ |\ i\ |\ s\ |\ \text{\code{\{\}}} && \text{values}\\
e & ::= v\ |\ x\ |\ e + e\ |\ x.p\ |\ x[i]\ && \text{expressions}\\
s & ::= \code{if}(x)\ s\ \code{else}\ s\ |\ x = e\ |\ x.p = e\ |\ x[i] = e && \text{statements}\\[0.7ex]
\tau & ::= \code{undefined}\ |\ \code{int}\ |\ \code{number}\ |\ \code{string}\ |\ o && \text{types}\\
T & ::= \mathcal{P}(\tau) && \text{type sets}\\[0.8ex]
C & ::= T \supseteq T\ |\ T \barrier T && \text{constraints}
\end{align*}
\caption{Simplified JavaScript Core, Types, and Constraints}
\label{fig:js-core}
\end{figure}


%\Section\ref{sec:object-types} describes object types in more detail,
%\Section\ref{sec:constraints} describes generation of the type constraints
%which forms the static portion of the analysis and the semantic triggers
%these require at runtime, and \Section\ref{sec:analysis-barriers}
%describes the use of type barriers to improve analysis precision.

\subsection{Object Types}
\label{sec:object-types}

To reason about the effects of property accesses, we need type information
for JavaScript objects and their properties.
Each object is immutably assigned an {\it object type} $o$.
When $o \in T_e$ for some expression $e$, then the possible values
for $e$ when it is executed include all objects with type $o$.

For the sake of brevity and ease of exposition, our simplified JavaScript core
only contains the ability to construct \code{Object}-prototyped object
literals via the \code{\{\}} syntax;
two objects have the same type when they were allocated via the same literal.
\footnote{
In full JavaScript, types are assigned to objects according to their prototype:
all objects with the same type have the same prototype.
Additionally, objects with the same prototype have the same type,
except for plain \code{Object}, \code{Array} and \code{Function} objects.
\code{Object} and \code{Array} objects have the same type if they were
allocated at the same source location,
and \code{Function} objects have the same type if they are closures
for the same script.
\code{Object} and \code{Function} objects which represent builtin objects
such as class prototypes, the \code{Math} object and native functions
are given unique types, to aid later optimizations
(\Section\ref{sec:definite-properties}).}

The type of an object is nominal: it is independent from the properties it
has. Objects which are structurally identical may have different types, and
objects with the same type may have different structures. This is crucial for
efficient analysis. JavaScript allows addition or deletion of object properties
at any time. Using structural typing would make an object's type a
flow-sensitive property.

Instead, for each object type we compute the possible properties which
objects of that type can have and the possible types of those properties.
These are denoted as type sets $\mathit{prop}(o,p)$ and
$\mathit{index}(o)$. The set $\mathit{prop}(o,p)$ captures the possible types
of a non-integer property $p$ for objects with type $o$, while
$\mathit{index}(o)$ captures the possible types of all integer properties of
all objects with type $o$.
These sets cover the types of both ``own'' properties (those directly held
by the object) as well as properties inherited from the object's prototype.
%We do not show how types are propagated from a prototype's type information
%to the instances; since all objects of a type share the same prototype,
%this transfer is straightforward.

\subsection{Type Constraints}
\label{sec:constraints}

\begin{figure}
\begin{align*}
& \inferrule{}{\judge^e \code{undefined} : T_{\code{u}}} &&
\left\{
\begin{array}{l}
T_{\code{u}} \supseteq \{\code{undefined}\}
\end{array}
\right\}
\tag{\sc Undef}\\
& \inferrule{}{\judge^e i : T_i} &&
\left\{
\begin{array}{l}
T_i \supseteq \{\code{int}\}
\end{array}
\right\} \tag{\sc Int}\\
& \inferrule{}{\judge^e s : T_s} &&
\left\{
\begin{array}{l}
T_s \supseteq \{\code{string}\}
\end{array}
\right\} \tag{\sc Str}\\
& \inferrule{}{\judge^e \text{\code{\{\}}} : T_{\text{\code{\{\}}}}} &&
\left\{
\begin{array}{l}
T_{\text{\code{\{\}}}} \supseteq \{o\}
\end{array}
\right\} \text{ where $o$ fresh} \tag{\sc Obj}\\
& \inferrule{}{\judge^e x : T_x} && \emptyset \tag{\sc Var}\\
& \inferrule{\judge^e x : T_x ~~~ \judge^e y : T_y}{\judge^e x + y : T_{x+y}} \\
& \mathrlap{\quad \left\{
\begin{array}{l}
T_{x+y} \supseteq \{\code{int}\}\ |\ \code{int} \in T_x \wedge \code{int} \in T_y,\\
T_{x+y} \supseteq \{\code{number}\}\ |\ \code{int} \in T_x \wedge \code{number} \in T_y,\\
T_{x+y} \supseteq \{\code{number}\}\ |\ \code{number} \in T_x \wedge \code{int} \in T_y,\\
T_{x+y} \supseteq \{\code{string}\}\ |\ \code{string} \in T_x \vee \code{string} \in T_y
\end{array}
\right\}} \tag{\sc Add}\\
& \inferrule{\judge^e x : T_x}{\judge^e x.p : T_{x.p}} &&
\left\{
\begin{array}{l}
T_{x.p} \barrier \mathit{prop}(o,p)\ |\ o \in T_x
\end{array}
\right\} \tag{\sc Prop}\\
& \inferrule{\judge^e x : T_x}{\judge^e x[i] : T_{x[i]}} &&
\left\{
\begin{array}{l}
T_{x[i]} \barrier \mathit{index}(o)\ |\ o \in T_x
\end{array}
\right\} \tag{\sc Index}\\
%\rulename{App} & \inferrule{\judge^e f : T_f ~~~ \judge^e x : T_x}{\judge^e f(x) : T_{f(x)}} \quad
%\left\{
%\begin{array}{l}
%T_{\mathit{dom}(f)} \supseteq T_x\ |\ T_{\mathit{dom}(f)} \rightarrow T_{\mathit{cod}(f)} \in T_f,\\
%T_{f(x)} \supseteq T_{\mathit{cod}(f)}\ |\ T_{\mathit{dom}(f)} \rightarrow T_{\mathit{cod}(f)} \in T_f
%\end{array}
%\right\}\\
& \inferrule{\judge^e x : T_x ~~~ \judge^e e : T_e}{\judge^s x = e : \text{\textbullet}} &&
\left\{
\begin{array}{l}
T_x \supseteq T_e
\end{array}
\right\} \tag{\sc A-Var}\\
& \inferrule{\judge^e x : T_x ~~~ \judge^e e : T_e}{\judge^s x.p = e : \text{\textbullet}} &&
\left\{
\begin{array}{l}
\mathit{prop}(o,p) \supseteq T_e\ |\ o \in T_x
\end{array}
\right\} \tag{\sc A-Prop}\\
& \inferrule{\judge^e x : T_x ~~~ \judge^e e : T_e}{\judge^s x[i] = e : \text{\textbullet}} &&
\left\{
\begin{array}{l}
\mathit{index}(o) \supseteq T_e\ |\ o \in T_x
\end{array}
\right\} \tag{\sc A-Index}\\
& \mathrlap{\inferrule{}{\judge^s \code{if}(x)\ s_1\ \code{else}\ s_2 : \text{\textbullet}} \quad
\mathcal{C}_s(s_1) \cup \mathcal{C}_s(s_2)} \tag{\sc If}
\end{align*}
\caption{Constraint Generation Rules}
\label{fig:constraint-rules}
\end{figure}

The static portion of our analysis generates constraints
modeling the flow of types
through the program. We assign to each
expression a type set representing the set of types it may have at runtime.
These constraints are
unsound with respect to JavaScript semantics. Each constraint is augmented
with triggers to fill in the remaining possible behaviors of the
operation.
For each rule, we informally describe the required triggers.

The grammar of constraints are shown in Figure~\ref{fig:js-core}. We have the
standard subset constraint, $\supseteq$, and a \emph{barrier subset
  constraint}, styled $\barrier$. For two type sets $X$ and $Y$, $X \supseteq
Y$ means that all types in $Y$ are propagated to $X$. On the other hand, $X
\barrier Y$ means that if $Y$ contains types that are not in $X$, then a
type barrier is required which updates the types in $X$
according to values which are dynamically assigned to the location
$X$ represents (\Section\label{sec:analysis-barriers}).

Rules for the constraint generation functions, $\mathcal{C}_e(e)$ for
expressions (styled $\judge^e$) and $\mathcal{C}_s(s)$ for statements (styled $\judge^s$),
are shown in
Figure~\ref{fig:constraint-rules}. Statically analyzing a function takes
the union of the results from applying $\mathcal{C}_s$ to every
statement in the method.

The \textsc{Undef}, \textsc{Int}, \textsc{Str}, and \textsc{Obj}
rules for literals and the \textsc{Var} rule for variables are
straightforward.

The \textsc{Add} rule is complex, as addition in JavaScript is similarly complex. It is
defined for any combination of values, can perform either a numeric addition,
string concatenation, or even function calls if either of its operands is an
object (calling their \code{valueOf} or \code{toString} members, producing a
number or string).

Using incomplete modeling lets us cut through this complexity.
Additions in actual programs are typically used to add two numbers
or concatenate a string with something else.
We statically model exactly these cases
and use semantic triggers to monitor the results produced by other
combinations of values, at little runtime cost.
Note that even the integer addition rule we have given is incomplete: the
result will be marked as an integer, ignoring the possibility of
overflow.
%When addition of two integers overflows, the result is not
%expressible in 32 bits, and has type \code{number}.
%Integer overflows are very rare, and as described in \Section\ref{sec:example}
%we treat integer additions as producing integers
%and use a semantic trigger to dynamically update the result of
%overflowing additions.

\textsc{Prop} accesses a named property $p$ from the possible objects referred
to by $x$, with the result the union of $\mathit{prop}(o,p)$ for all such
objects.  This rule is complete only in cases where the object referred to by
$x$ (or its prototype) actually has the $p$ property. Accesses on properties
which are not actually part of an object produce \code{undefined}.
Accesses on missing properties are rare,
and yet in many cases we cannot prove that an object
definitely has some property. In such cases we do not dilute the
resulting type sets with \code{undefined}. We instead use a trigger
on execution paths accessing a missing property to update the result type
of the access with \code{undefined}.

\textsc{Index} is similar to \textsc{Prop}, with the added problem that any
property of the object could be accessed.  In JavaScript, \code{x["p"]} is
equivalent to \code{x.p}. If $x$ has the object type $o$, an index operation
can access a potentially infinite number of type sets $\mathit{prop}(o,p)$.
Figuring out exactly which such properties are possible is generally
intractable. We do not model such arbitrary accesses at all, and treat
all index operations as operating on an integer,
 which we collapse into a single type set
$\mathit{index}(o)$.  In full JavaScript, any indexed access which is on a
non-integer property, or is on an integer property which is missing from an
object, must be accounted for with triggers in the same manner as
\textsc{Prop}.

\textsc{A-Var}, \textsc{A-Prop} and \textsc{A-Index} invert the
corresponding read expressions.  These rules are complete, except that
\textsc{A-Index} presumes that an integer property is being accessed.
Again, in full JavaScript, the effects on $\mathit{prop}(o,p)$ resulting from
assignments to a string index \code{x["p"]} on some $x$ with object type $o$
must be accounted for with runtime checks.

Our analysis is flow-insensitive, so the \textsc{If} rule is simply the union
of the constraints generated by the branches.

\subsection{Type Barriers}
\label{sec:analysis-barriers}

As described in \Section\ref{sec:example}, type barriers are dynamic type checks
inserted to improve analysis precision in the presence
of polymorphic code.
Propagation along an assignment $X = Y$ can be modeled
statically as a subset
constraint $X \supseteq Y$ or dynamically as a barrier constraint
$X \barrier Y$.
It is always safe to use one in place of the other; in \Section\ref{sec:without_barriers}
we show the effect of always using subset constraints in lieu of
barrier constraints.

For a barrier constraint $X \barrier Y$, a type barrier is required whenever
$X \not\supseteq Y$. The barrier dynamically checks that the type of each value
flowing across the assignment is actually in $X$, and updates $X$ whenever
values of a new type are encountered.
Thought of another way, the vanilla subset constraint propagates
all types at analysis time. The barrier subset constraint does not propagate
types at analysis time but defers with dynamic checks, propagating the types
only if necessary during runtime.

Type barriers are much like dynamic type casts in Java: assignments from a
more general type to a more specific type are possible as long as a
dynamic test occurs for conformance.
However, rather than throw an exception (as in Java) a tripped type barrier will
despecialize the target of the assignment.

The presence or absence of type barriers for a given barrier constraint is not
monotonic with respect to the contents of the type sets in the program.  As
new types are discovered, new type barriers may be required, and existing ones
may become unnecessary.  However, it is always safe to perform the runtime
tests for a given barrier.

Recall our hypothetical situation from \Section\ref{sec:example} where \code{Box} is
used as a polymorphic structure containing either an integer or a string in
the example program from Figure~\ref{fig:motivating-example}. The subset
barrier constraint on line 8 is $T_{\code{a[i]}} \barrier T_{\code{Box}}$,
with $T_{\code{a[i]}} = \{\code{int}\}$ and $T_{\code{Box}} = \{\code{int},
\code{string}\}$. Since $T_{\code{a[i]}} \not\supseteq T_{\code{Box}}$, a type
barrier is required.

In the constraint generation rules in Figure~\ref{fig:constraint-rules} we
present two rules which employ type barrers:
\textsc{Prop}, and \textsc{Index}. In practice, we also use type barriers for
call argument binding to precisely model polymorphic call sites where
only certain combinations of argument types and callee functions are possible.
Barriers could be used for other types of assignments, but we do not do so.
Allowing barriers in new places is unlikely to significantly change the total
number of required barriers --- improving precision by adding barriers in one
place can make barriers in another place unnecessary.
%The main cost is the extra compiler complexity required to support new
%kinds of barriers, and we may take such steps in the future.

\subsection{Supplemental Analyses}
\label{sec:supplemental-analyses}

In many cases type information itself is insufficient to generate code
which performs comparably to a statically-typed language such as Java.
Semantic triggers are generally cheap, but they nevertheless incur a cost.
%Integer addition requires checks for overflow,
%index operations on arrays require checks for holes in the middle of the array,
%and property accesses require determining whether the object has the property
%and the physical location where the property is stored.
These checks should be eliminated in as many cases as possible.

Eliminating such checks requires more detailed analysis information.
Rather than build additional complexity into the type analysis itself,
we use supplemental analyses which leverage type information but do not
modify the set of inferred types.
%The remainder of this section briefly describes our handling of the above
%situations.
We do several other supplemental analyses, but those described below are the most important.

\paragraph{Integer Overflow}

In the execution of a JavaScript program, the overall cost of doing integer
overflow checks is very small.
On kernels which do many additions, however, the cost can become significant.
We have measured overflow check overhead at 10-20\% of total execution
time on microbenchmarks.

Using type information, we normally know statically where integers are being
added. We use two techniques on those sites to remove overflow checks.
First, for simple additions in a loop (mainly loop counters) we try to use
the loop termination condition to compute a range
check which can be hoisted from the loop, a standard technique
which can only be performed for JavaScript with type information available.
Second, integer additions which are used as inputs to bitwise operators
do not need overflow checks, as bitwise operators truncate their inputs to 32
bit integers.

\paragraph{Packed Arrays}

Arrays are usually constructed by writing to their elements in ascending
order, with no gaps; we call these arrays {\it packed}.
Packed arrays do not have holes in the middle, and if an access is statically
known to be on a packed array then only a bounds check is required.
There are a large number of ways packed arrays can be constructed, however,
which makes it difficult to statically prove an array is packed.
Instead, we dynamically detect out-of-order writes on an array,
and mark the type of the array object as possibly not packed.
If an object type has never been marked as not packed, then all objects
with that type are packed arrays.

%When compiling an index operation, if the type set of the accessed object
%contains no object types marked as possibly not packed, then the operation
%must be on a packed array.
%In such cases, the generated code does not need to check for a hole in the
%middle of the array, and only needs to perform a bounds check. In turn, the bounds check can
%often be hoisted from loop bodies in the same manner as integer overflow
%checks.

The packed status of an object type can change dynamically due to out-of-order
writes, possibly invalidating JIT code.
%This forces recompilation in the same way as type changes due to semantic
%triggers and tripped type barriers do.

\paragraph{Definite Properties}
\label{sec:definite-properties}

JavaScript objects are internally laid out as a map from property names
to slots in an array of values.
If a property access can be resolved statically to a particular slot in the
array, then the access is on a {\it definite} property and can be compiled
as a direct lookup. This is comparable to
field accesses on monomorphic receivers in a language with static object layouts,
such as Java or C++.

We identify definite property accesses in three ways.
First, if the property access is on an object with a unique type,
we know the exact JavaScript object being accessed and can use the slot
in its property map.
Second, object literals allocated in the same place have the same type,
and definite properties can be picked up from the order the literal
adds properties.
Third, objects created by calling \code{new} on the same function will have the
same prototype (unless the function's \code{prototype} property is overwritten),
and we analyze the function's body to identify properties it definitely
adds before letting the new object escape.

These techniques are sensitive to properties being deleted or
reconfigured, and if such events happen then JIT code will be invalidated
in the same way as by packed array or type set changes.

%%% Local Variables: 
%%% mode: latex
%%% TeX-master: "main"
%%% End: 


\section{Implementation}
\label{sec:implementation}

We have implemented this analysis for SpiderMonkey, the Java-Script engine
in Firefox.
We have also modified the engine's JIT compiler, JaegerMonkey, to use
inferred type information when generating code.
Without type information, JaegerMonkey generates code in a fairly mechanical
translation from the original SpiderMonkey bytecode for a script.
Using type information, we were able to improve on this in several ways:

\begin{itemize}

\item Values with statically known types can be tracked in JIT-compiled code
using an untyped representation.
Encoding the type in a value requires significant memory traffic or
marshaling overhead.
%Encoding the type in a value requires significant memory traffic
%if the type is stored separately from the actual data in a value
%(as it is in SpiderMonkey),
%or marshaling overhead if a single word is used to encode values.
An untyped representation stores just the data component of a value.
%a raw pointer for objects and strings, and raw bits for other value types ---
%without explicitly encoding the type.
Additionally, knowing the type of a value statically eliminates many
dynamic type tests.
%which the code would otherwise require.

\item Several classical compiler optimizations were added,
including linear scan register allocation, loop invariant code motion
and function call inlining.

These optimizations could be applied without
having static type information.
Doing so is, however, far more difficult and far less effective than in
the case where types are known.\footnote{For example, loop invariant code motion depends
on knowing whether operations
are idempotent, while in general JavaScript operations are not,
and register allocation requires types to determine whether values should
be stored in general purpose or floating point registers.}

\end{itemize}

In \Section\ref{sec:recompilation} we describe how we handle dynamic recompilation in
response to type changes, and in \Section\ref{sec:memory} we describe the techniques
used to manage analysis memory usage.

%
% I don't understand figure placement. Why do I have to put this here to have
% it appear on the page I want?
%

\begin{figure*}[ht]
\centering
\begin{tabular}{lrrrrrrrrrrr}

\toprule

           & \multicolumn{2}{c}{JM Compilation}
           & \multicolumn{2}{c}{JM+TI Compilation}
           & %\multicolumn{2}{c}{}
           & \multicolumn{3}{c}{$\times$1 Times (ms)}
           & \multicolumn{3}{c}{$\times$20 Times (ms)} \\

\cmidrule(r){2-3}
\cmidrule{4-5}
\cmidrule(r){7-9}
\cmidrule{10-12}

Test       & Time (ms) & \#
           & Time (ms) & \#
           & Ratio
           & JM
           & JM+TI
           & Ratio
           & JM
           & JM+TI
           & Ratio \\

\midrule

3d-cube                  & 2.68 & 15 & 8.21 & 24 & 3.06
    & 14.1 & 16.6 & 1.18 & 226.9 & 138.8 & 0.61 \\
3d-morph                 & 0.55 & 2   & 1.59 & 7  & 2.89
    & 9.8  & 10.3 & 1.05 & 184.7 & 174.6 & 0.95 \\
3d-raytrace              & 2.25 & 19 & 6.04 & 22 & 2.68
    & 14.7 & 15.6 & 1.06 & 268.6 & 152.2 & 0.57 \\
access-binary-trees      & 0.63 & 4   & 1.03 & 7  & 1.63
    & 6.1  & 5.2 & 0.85   & 101.4 & 70.8  & 0.70 \\
access-fannkuch          & 0.65  & 1  & 2.43 & 4  & 3.76
    & 15.3 & 10.1 & 0.66  & 289.9 & 113.7 & 0.39 \\
access-nbody             & 1.01 & 5  & 1.49 & 5  & 1.47
    & 9.9  & 5.3 & 0.54   & 175.6 & 73.2  & 0.42 \\
access-nsieve            & 0.28 & 1   & 0.63 & 2   & 2.25
    & 6.9  & 4.5 & 0.65   & 143.1 & 90.7  & 0.63 \\
bitops-3bit-bits-in-byte & 0.28 & 2   & 0.58 & 3   & 2.07
    & 1.7  & 0.8 & 0.47    & 29.9 & 10.0   & 0.33 \\
bitops-bits-in-byte      & 0.29 & 2   & 0.54 & 3   & 1.86
    & 7.0  & 4.8 & 0.69   & 139.4 & 85.4  & 0.61 \\
bitops-bitwise-and       & 0.24 & 1   & 0.39 & 1   & 1.63
    & 6.1  & 3.1 & 0.51   & 125.2 & 63.7  & 0.51 \\
bitops-nsieve-bits       & 0.35 & 1   & 0.73 & 2   & 2.09
    & 6.0  & 3.6 & 0.60    & 116.1 & 63.9  & 0.55 \\
controlflow-recursive    & 0.38 & 3   & 0.65 & 6   & 1.71
    & 2.6  & 2.7 & 1.04  & 49.4  & 42.3  & 0.86 \\
crypto-aes               & 2.04 & 14 & 6.61 & 23 & 3.24
    & 9.3  & 10.9 & 1.17 & 162.6 & 107.7 & 0.66 \\
crypto-md5               & 1.81 & 9  & 3.42 & 13 & 1.89
    & 6.1  & 6.0 & 0.98   & 62.0  & 27.1  & 0.44 \\
crypto-sha1              & 0.88 & 7   & 2.46 & 11 & 2.80
    & 3.1  & 4.0 & 1.29  & 44.2  & 19.4  & 0.44 \\
date-format-tofte        & 0.93 & 21  & 2.27 & 24 & 2.44
    & 16.4 & 18.3 & 1.12 & 316.6 & 321.8 & 1.02 \\
date-format-xparb        & 0.88 & 7   & 1.26 & 6  & 1.43
    & 11.6 & 14.8 & 1.28 & 219.4 & 285.1 & 1.30 \\
math-cordic              & 0.45 & 3   & 0.94 & 5   & 2.09
    & 7.4  & 3.4 & 0.46   & 141.0 & 50.3  & 0.36 \\
math-partial-sums        & 0.47 & 1   & 1.03 & 3  & 2.19
    & 14.1 & 12.4 & 0.88  & 278.4 & 232.6 & 0.84 \\
math-spectral-norm       & 0.54 & 5   & 1.39 & 9  & 2.57
    & 5.0  & 3.4 & 0.68   & 92.6  & 51.2  & 0.55 \\
regexp-dna               & 0.00 & 0   & 0.00 & 0    & 0.00
    & 16.3 & 16.1 & 0.99  & 254.5 & 268.8 & 1.06 \\
string-base64            & 0.87 & 3   & 1.90 & 5   & 2.18
    & 7.8  & 6.5 & 0.83   & 151.9 & 103.6 & 0.68 \\
string-fasta             & 0.59 & 4   & 1.70 & 9   & 2.88
    & 10.0 & 7.3 & 0.73   & 124.0 & 93.4  & 0.75 \\
string-tagcloud          & 0.54 & 4   & 1.54 & 6  & 2.85
    & 21.0 & 24.3 & 1.16 & 372.4 & 433.4 & 1.17 \\
string-unpack-code       & 0.89 & 8   & 2.65 & 16 & 2.98
    & 24.4 & 26.7 & 1.09 & 417.6 & 442.5 & 1.06 \\
string-validate-input    & 0.58 & 4   & 1.65 & 8  & 2.84
    & 10.2 & 9.5 & 0.93   & 216.6 & 184.1 & 0.85 \\

\midrule

Total                  & 21.06 & 146 & 53.13 & 224 & 2.52
    & 261.9 & 246.4 & 0.94 & 4703.6 & 3700.3 & 0.79 \\

\bottomrule

\end{tabular}
\nocaptionrule \caption{SunSpider-0.9.1 Benchmark Results}
\label{fig:sunspider}
\end{figure*}

\subsection{Recompilation}
\label{sec:recompilation}

As described in \Section\ref{sec:example}, computed type information can change as
a result of runtime checks, newly analyzed code or other dynamic behavior.
For compiled code to rely on this type information, we must be able
to recompile the code in response to changes in types while that code is
still running.

%Compiling a function will only query type information, and will not change it.
%The type sets queried can be for any expression %or lvalue
%in the program,
%independent of the function being compiled,
%and the dependencies of a compiled function need to be encoded efficiently
%and precisely, to avoid unnecessary recompilation.

As each script is compiled, we keep track of all type information queried
by the compiler.
Afterwards, the dependencies are encoded and attached
to the relevant type sets,
and if those type sets change in the future the script is
marked for recompilation.
We represent the contents of type sets explicitly and eagerly resolve
constraints, so that new types immediately
trigger recompilation with little overhead.

%When starting this project, the best strategy for recompilation
%was not apparent.
%We originally tried to immediately generate new JIT code for a script,
%scan the stack for calls made from the script's old code and patch the
%associated return addresses with the corresponding addresses in the
%new code.
%This is a simple design and easy to implement, but had a few major problems:

%\begin{itemize}

%\item When recompiling a script, the compiler had to be sure to generate
%a superset of the calls which appeared in the original script.
%This places complicated constraints on the optimizations that
%can be done when compiling or recompiling.

%\item If many type sets change within a short time period (but not all at
%the same time), the recompiler could thrash and repeatedly recompile a
%method, incurring large overhead.

%\end{itemize}

%To address these, we changed the design so that return addresses of
%calls made by the old code were replaced with the addresses of
%a few specialized trampolines.
%The JIT code for the script is discarded, and the script is not
%immediately recompiled.
%After control returns to the trampolines, they use metadata added to
%the calling frame to assemble
%a valid state for the JavaScript engine's bytecode interpreter,
%and resume execution in the interpreter.

%Immediate recompilation causes problems such as frequent recompilation
%(thrashing) and complex constraints on optimizations. The details of these
%problems are elided for brevity.
When a script is marked for recompilation, we
discard the JIT code for the script, and resume execution in the
interpreter.
We do not compile scripts until after a certain number of calls or loop back
edges are taken, and these counters are reset whenever discarding JIT code.
Once the script warms back up, it will be recompiled using the new type information
in the same manner as its initial compilation.
%Since the initial compilation and recompilations occur independently
%from one another, the compiler is free to implement any optimization it
%wants provided that a valid interpreter state can always be recovered ---
%only information which is provably dead (regardless of currently known types)
%can be discarded.
%In practice this restriction has not been a problem for
%implementing optimizations.
%Since the function will execute in the interpreter some before
%being recompiled, thrashing is reduced, and thresholds can be adjusted to
%increase the amount of time spent interpreting if thrashing is detected
%dynamically.

\subsection{Memory Management}
\label{sec:memory}

Two major goals of JIT compilation in a web browser stand in stark contrast
to one another: generate code that is as fast as possible, and use as little
memory as possible.
JIT code can consume a large amount of memory, and the type sets and constraints
computed by our analysis consume even more.
We reconcile this conflict by observing how browsers are used in practice:
to surf the web.
The web page being viewed, content being generated, and JavaScript code being
run are constantly changing.
The compiler and analysis need to not only quickly adapt to new scripts that are
running, but also to quickly discard regenerable data associated with
old scripts that are no longer running much, even if the old scripts are still
reachable and not subject to garbage collection.

We do this with a simple trick:
on every garbage collection, we throw away all JIT code and as much analysis
information as possible.
All inferred types are functionally determined from a small core of type
information:
type sets for the properties of objects, function arguments, the observed
type sets associated with barrier constraints and the semantic triggers which
have been tripped.
All type constraints and all other type sets are discarded, notably the type sets
describing the intermediate expressions in a function without barriers on them.
This constitutes the great majority of the memory allocated for analysis.
Should the involved functions warm back up and require recompilation,
they will be reanalyzed. In combination with the retained type information,
the complete analysis state for the function is then recovered.

In Firefox, garbage collections typically happen every several seconds.
If the user is quickly changing pages or tabs, unused JIT code and analysis
information will be quickly destroyed.
If the user is staying on one page, active scripts may be repeatedly
recompiled and reanalyzed, but the timeframe between collections keeps this
as a small portion of overall runtime.
When many tabs are open (the case where memory usage is most important
for the browser), analysis information typically accounts
for less than 2\% of the browser's overall memory usage.

%%% Local Variables: 
%%% mode: latex
%%% TeX-master: "main"
%%% End: 

\section{Evaluation}
\label{sec:evaluation}

We evaluate the effectiveness of our analysis in two ways.
In \Section\ref{sec:benchmarks} we compare the performance on major JavaScript benchmarks
of a single compiler with and without use of analyzed type information.
In \Section\ref{sec:websites} we examine the behavior of the analysis on a selection of
websites which heavily use JavaScript to gauge analysis effectiveness in practice.

\subsection{Benchmark Performance}
\label{sec:benchmarks}


As described in \Section\ref{sec:implementation}, we have integrated our analysis into
the Jaegermonkey JIT compiler used in Firefox.
We compare performance of the compiler used both without the analysis (JM)
and with the analysis (JM+TI).
JM+TI adds several major optimizations to JM,
and requires additional compilations due to dynamic type changes
(\Section\ref{sec:recompilation}).
Figure~\ref{fig:sunspider} shows the effect of these changes on the popular
SunSpider JavaScript benchmark\footnote{\url{http://www.webkit.org/perf/sunspider/sunspider.html}}.

The JM and JM+TI sections of Figure~\ref{fig:sunspider} show the total amount of time
spent compiling and the total number of compilations for both versions of
the compiler.
For JM+TI, compilation time also includes time spent generating and solving
type constraints, which is small: 4ms for the entire benchmark.
JM performs 146 compilations, while JM+TI performs 224, an increase of 78.
The total compilation time for JM+TI is 2.52 times that of JM, an increase
of 32ms, due a combination of recompilations, type analysis and the extra
complexity of the added optimizations.

Despite the significant extra compilation cost, the type-based optimizations
performed by JM+TI quickly pay for themselves. The $\times$1 and
$\times$20 sections
of Figure~\ref{fig:sunspider} show the running times of the two versions of the
compiler on the benchmark run once and modified to run twenty times,
respectively.
In the single run case JM+TI is a 6.3\% improvement over JM.
One run of SunSpider completes in less than 250ms, which makes it
difficult to get an optimization to pay for itself on this benchmark.
JavaScript heavy webpages are typically viewed for longer than
1/4 of a second, and longer execution times better show the effect
of type based optimizations.
When run twenty times, the speedup given by JM+TI increases to 27.1\%.

Figures~\ref{fig:v8bench} and~\ref{fig:kraken} compare the performance of JM and JM+TI
on two other popular benchmarks, the V8\footnote{\url{http://v8.googlecode.com/svn/data/benchmarks/v6/run.html}}
and Kraken\footnote{\url{http://krakenbenchmark.mozilla.org}} suites.
These suites run for several seconds each, far longer than SunSpider,
and show a larger speedup.
V8 scores (which are given as a rate, rather than a raw time; larger is better) improve by 50\%,
and Kraken scores improve by a factor of 2.69.

Across the benchmarks, not all tests improved equally, and some regressed over
the engine's performance without the analysis.
These include the date-format-xparb and string-tagcloud tests in SunSpider,
and the RayTrace and RegExp tests in the V8.
These are tests which spend little time in JIT code, and perform many side
effects in VM code itself.
Changes to objects which happen in the VM due to, e.g., the behavior of
builtin functions must be tracked to ensure the correctness of type
information for the heap.
We are working to reduce the overhead incurred by such side effects.

\subsubsection{Performance Cost of Barriers}
\label{sec:barrier_cost}

The cost of using type barriers is of crucial importance for two reasons.
First, if barriers are very expensive then the effectiveness of the compiler
on websites which require many barriers (\Section\ref{sec:barriers})
is greatly reduced.
Second, if barriers are very cheap then the time and memory spent
tracking the types of heap values would be unnecessary.

To estimate this cost, we modified the compiler to artificially introduce
barriers at every indexed and property access, as if the types of all values
in the heap were unknown.
For benchmarks, this is a great increase above the baseline barrier
frequency (\Section\ref{sec:barriers}).
Figure~\ref{fig:benchmarks100} gives times for the modified compiler on the
tracked benchmarks.
On a single run of SunSpider, performance was even with the JM
compiler.
In all other cases, performance was significantly better than the JM
compiler and significantly worse than the JM+TI compiler.

This indicates that while the compiler will still be able to effectively
optimize code in cases where types of heap values are not well known,
accurately inferring such types and minimizing the barrier count is important
for maximizing performance.

\begin{figure}
\centering
\begin{tabular}{lrrrr}
\toprule
Test & JM & JM+TI & Ratio \\
\midrule
Richards & 4497 & 7152 & 1.59 \\
DeltaBlue & 3250 & 9087 & 2.80 \\
Crypto & 5205 & 13376 & 2.57 \\
RayTrace & 3733 & 3217 & 0.86 \\
EarleyBoyer & 4546 & 6291 & 1.38 \\
RegExp & 1547 & 1316 & 0.85 \\
Splay & 4775 & 7049 & 1.48 \\
\midrule
Total & 3702 & 5555 & 1.50 \\
\bottomrule
\end{tabular}
\nocaptionrule \caption{V8 (version 6) Benchmark Scores (higher is better)}
\label{fig:v8bench}
\end{figure}

\begin{figure}
\centering
\begin{tabular}{lrrr}
\toprule
Test & JM (ms) & JM+TI (ms) & Ratio \\
\midrule
ai-astar & 889.4 & 137.8 & 0.15 \\
audio-beat-detection & 641.0 & 374.8 & 0.58 \\
audio-dft & 627.8 & 352.6 & 0.56 \\
audio-fft & 494.0 & 229.8 & 0.47 \\
audio-oscillator & 518.0 & 221.2 & 0.43 \\
imaging-gaussian-blur & 4351.4 & 730.0 & 0.17 \\
imaging-darkroom & 699.6 & 586.8 & 0.84 \\
imaging-desaturate & 821.2 & 209.2 & 0.25 \\
json-parse-financial & 116.6 & 119.2 & 1.02 \\
json-stringify-tinderbox & 80.0 & 78.8 & 0.99 \\
crypto-aes & 201.6 & 158.0 & 0.78 \\
crypto-ccm & 127.8 & 133.6 & 1.05 \\
crypto-pbkdf2 & 454.8 & 350.2 & 0.77 \\
crypto-sha256-iterative & 153.2 & 106.2 & 0.69 \\
\midrule
Total & 10176.4 & 3778.2 & 0.37 \\
\bottomrule
\end{tabular}
\nocaptionrule \caption{Kraken-1.1 Benchmark Results}
\label{fig:kraken}
\end{figure}

\begin{figure}
\centering
\begin{tabular}{lrrr}
\toprule
Suite & Time/Score & vs. JM & vs. JM+TI \\
\midrule
Sunspider-0.9.1 $\times$1  & 262.2  & 1.00 & 1.06 \\
Sunspider-0.9.1 $\times$20 & 4044.3 & 0.86  & 1.09 \\
Kraken-1.1          & 7948.6 & 0.78  & 2.10 \\
V8 (version 6)      & 4317   & 1.17 & 0.78 \\
\bottomrule
\end{tabular}
\nocaptionrule \caption{Benchmark Results with 100\% barriers}
\label{fig:benchmarks100}
\end{figure}

\begin{figure*}
\centering
\begin{tabular}{lrrrrrrrrrrrrrrr}
\toprule
& \multicolumn{3}{c}{Inferred Precision (\%)} &
& \multicolumn{4}{c}{Arithmetic (\%)}
& \multicolumn{4}{c}{Indices (\%)} \\

\cmidrule(r){2-4}
\cmidrule(r){6-9}
\cmidrule{10-13}

Test
& Mono & Di & Poly & Barrier (\%)
& Int & Double & Other & Unknown
& Int & Double & Other & Unknown \\
\midrule
gmail          & 78 & 5  & 17 & 47 & 62 & 9  & 7  & 21 & 44 & 0 & 47 & 8 \\
googlemaps     & 81 & 7  & 12 & 36 & 66 & 26 & 3  & 5  & 60 & 6 & 30 & 4 \\
facebook       & 73 & 11 & 16 & 42 & 43 & 0 & 40 & 16 & 62 & 0 & 32 & 6 \\
flickr         & 71 & 19 & 10 & 74 & 61 & 1 & 30 & 8 & 27 & 0 & 70 & 3 \\
grooveshark    & 64 & 15 & 21 & 63 & 65 & 1 & 13 & 21 & 28 & 0 & 56 & 16 \\
meebo          & 78 & 11 & 10 & 35 & 66 & 9 & 18 & 8 & 17 & 0 & 34 & 49 \\
reddit         & 71 & 7 & 22 & 51 & 64 & 0 & 29 & 7 & 22 & 0 & 71 & 7 \\
youtube        & 83 & 11 & 6 & 38 & 50 & 27 & 19 & 4 & 33 & 0 & 38 & 29 \\
ztype          & 91 & 1 & 9 & 52 & 43 & 41 & 8 & 8 & 79 & 9 & 12 & 0 \\
280slides      & 79 & 3 & 19 & 64 & 48 & 51 & 1 & 0 & 6 & 0 & 91 & 2 \\
membench50     & 76 & 11 & 13 & 49 & 65 & 7  & 18 & 10 & 44 & 0 & 47 & 10 \\
\midrule
sunspider      & 99 & 0  & 1  & 7  & 72 & 21 & 7  & 0  & 95 & 0 & 4  & 1  \\
v8bench        & 86 & 7  & 7  & 26 & 98 & 1  & 0  & 0  & 100 & 0 & 0 & 0  \\
kraken         & 100 & 0 & 0  & 3  & 61 & 37 & 2  & 0  & 100 & 0 & 0 & 0  \\
\midrule
angrybirds     & 97 & 2 & 1 & 93 & 22 & 78 & 0 & 0 & 88 & 8 & 0 & 5 \\
gameboy        & 88 & 0 & 12 & 16 & 54 & 36 & 3 & 7 & 88 & 0 & 0 & 12 \\
bullet         & 84 & 0 & 16 & 92 & 54 & 38 & 0 & 7 & 79 & 20 & 0 & 1 \\
lights         & 97 & 1 & 2 & 15 & 34 & 66 & 0 & 1 & 95 & 0 & 4 & 1 \\
FOTN           & 98 & 1 & 1 & 20 & 39 & 61 & 0 & 0 & 96 & 0 & 3 & 0 \\
monalisa       & 99 & 1 & 0 & 4 & 94 & 3 & 2 & 0 & 100 & 0 & 0 & 0 \\
\midrule
Overall        & 84.7 & 5.7 & 9.8 & 41.4 & 58.1 & 25.7 & 10.0 & 6.2 & 63.2 & 1.7 & 27.0 & 7.7 \\
\bottomrule
\end{tabular}
\nocaptionrule \caption{Website Type Profiling Results}
\label{fig:polymorphism}
\end{figure*}

\subsection{Website Performance}
\label{sec:websites}

In this section we measure the precision of the analysis
on a variety of websites.
The impact of compiler optimizations is difficult to accurately
measure on websites due to confounding issues like differences
in network latency and other browser effects.
Since analysis precision directly ties into the quality of
generated code, it makes a good surrogate for optimization effectiveness.

We modified Firefox to track several precision metrics while running,
all of which operate at the granularity of individual operations.
%These metrics are generally useful for debugging performance faults,
%and will be released with a future version of Firefox.
A brief description of the websites used is below.
A full description of the tested websites and methodology used for each
is available in the appendix of the full version of the paper.

\begin{itemize}

\item Ten popular websites which use JavaScript extensively.
Each site was used for several minutes, exercising various features.

\item The membench50 suite\footnote{\url{http://gregor-wagner.com/tmp/mem50}},
a memory testing framework
which loads the front pages of 50 popular websites.

\item The three benchmark suites described in \Section\ref{sec:benchmarks}.

\item Six games and demos which are bound on JavaScript performance.
Each was used for several minutes or, in the case of non-interactive
demos, viewed to completion.

\end{itemize}

When developing the analysis and compiler we tuned behavior for the three
covered benchmark suites, as well as various websites.
Besides the benchmarks, no tuning work has been done for any of the
websites described here.

We address several questions related to analysis precision,
listed below. The answers to these sometimes differ significantly
across the different categories of websites.

\begin{enumerate}

\item How polymorphic are values read at access sites?  (\Section\ref{sec:access_polymorphism})

\item How often are type barriers required?  (\Section\ref{sec:barriers})

\item How polymorphic are performed operations?  (\Section\ref{sec:operations})

\item How polymorphic are the objects used at access sites?  (\Section\ref{sec:access_objects})

\item How important are type barriers?  (\Section\ref{sec:without_barriers})

\end{enumerate}



\subsubsection{Access Site Polymorphism}
\label{sec:access_polymorphism}

The degree of polymorphism used in practice is of utmost importance
for our analysis.
The analysis is sound and will always compute a lower bound on the possible
types that can appear at the various points in a program,
so the precision of the generated type information is limited for
access sites and operations which are polymorphic in practice.
We draw the following distinction:

\begin{description}

\item[Monomorphic] Sites that have only ever produced a single kind of value.
Two values are of the same kind if they are either primitives of the same
type or both objects with possibly different
object types.
Access sites containing objects of multiple types can often be optimized
just as well as sites containing objects of a single type, as long as
all the observed object types share common attributes (\Section\ref{sec:access_objects}).

\item[Dimorphic] Sites that have produced either strings or objects (but not both),
and also at most one of the \code{undefined}, \code{null} or a boolean value.
At such sites, even though multiple kinds are possible an untyped
representation can still be used,
as a single test on the unboxed form will determine the type.
The untyped representation of objects and strings are pointers,
whereas \code{undefined}, \code{null} and booleans are either $0$ or $1$.

\item[Polymorphic] Sites that have produced values of multiple kinds,
and compiled code must use a typed representation which keeps track of
the value's kind.

\end{description}

The inferred precision section of Figure~\ref{fig:polymorphism} shows the fractions
of dynamic indexed element and property reads which were at a site inferred
as producing
monomorphic, dimorphic, or polymorphic sets of values.
All these sites have type barriers on them, so the set of inferred types
is equivalent to the set of observed types.

The category used for a dynamic access is determined from the types
inferred at the time of the access.
Since the types inferred for an access site can grow as a program executes,
dynamic accesses at the same site can contribute to different
columns over time.

Averaged across pages, 84.7\% of reads were at monomorphic
sites, and 90.2\% were at monomorphic or dimorphic sites.
The latter figure is 85.9\% for websites, 97.3\% for benchmarks,
and 94.7\% for games and demos; websites are more polymorphic than games
and demos, but by and large behave in a monomorphic fashion.

\subsubsection{Barrier Frequency}
\label{sec:barriers}

Examining the frequency with which type barriers are required
gives insight to the precision of the model of the heap constructed by
the analysis.

The barrier section of Figure~\ref{fig:polymorphism} shows the frequencies of
indexed and property accesses on sampled pages which required a barrier.
Averaged across pages, barriers were required on 41.4\% of such accesses.
There is a large disparity between websites and other pages.
Websites were fairly homogenous, requiring barriers on between 35\%
and 74\% of accesses (averaging 50\%), while benchmarks,
games and demos were generally
much lower, averaging 13\% except for two outliers above 90\%.

The larger proportion of barriers required for websites indicates that
heap layouts and types tend to be more complicated for websites than for
games and demos.
Still, the presence of the type barriers themselves means that we
detect as monomorphic the very large proportion of access sites which are,
with only a small amount of barrier checking overhead incurred by the
more complicated heaps.

DISCUSS OUTLIERS

\subsubsection{Operation Precision}
\label{sec:operations}

The arithmetic and indices sections of Figure~\ref{fig:polymorphism} show the frequency
of inferred types for arithmetic operations and the index operand of
indexed accesses, respectively.
These are operations for which precise type information is crucial
for efficient compilation, and give a sense of the precision of type
information for operations which do not have associated type barriers.

In the arithmetic section, the integer, double, other, and unknown columns indicate,
respectively,
operations on known integers which give an integer result,
operations on integers or doubles which give a double result,
operations on any other type of known value,
and operations where at least one of the operand types is unknown.
Overall, precise types were found for 93.8\% of arithmetic operations,
including 90.2\% of operations performed by websites.
Comparing websites with other pages, websites tend to do far more
arithmetic on non-numeric values --- 16.8\% vs. 1.6\% ---
and considerably less arithmetic on doubles --- 14.8\% vs. 37.9\%.

In the indices section, the integer, double, other, and
unknown columns indicate, respectively, that the type of the index, i.e.,
the type of \code{i} in an expression such as \code{a[i]}, is known
to be an integer, a double, any other known type, or unknown. Websites tend to
have more unknown index types than both benchmarks and games.

\begin{figure}
\centering
\begin{tabular}{lrrrrrr}
\toprule
     & \multicolumn{3}{c}{Indexed Acc. (\%)}
     & \multicolumn{3}{c}{Property Acc. (\%)} \\
\cmidrule(r){2-4}
\cmidrule{5-7}
Test & Packed & Array & Uk
     & Def & PIC & Uk \\
\midrule
gmail          & 90 & 4 & 5 & 31 & 57 & 12 \\
googlemaps     & 92 & 1 & 7 & 18 & 77 & 5 \\
facebook       & 16 & 68 & 16 & 41 & 53 & 6 \\
flickr         & 27 & 0 & 73 & 33 & 53 & 14 \\
grooveshark    & 90 & 2 & 8 & 20 & 66 & 14 \\
meebo          & 57 & 0 & 43 & 40 & 57 & 3 \\
reddit         & 97 & 0 & 3 & 45 & 51 & 4 \\
youtube        & 100 & 0 & 0 & 32 & 49 & 19 \\
ztype          & 100 & 0 & 0 & 23 & 76 & 0 \\
280slides      & 88 & 12 & 0 & 23 & 56 & 21 \\
membench50     & 80 & 4 & 16 & 35 & 58 & 6 \\
\midrule
sunspider      & 93 & 6 & 1 & 81 & 19 & 0 \\
v8bench        & 7 & 93 & 0 & 64 & 36 & 0 \\
kraken         & 99 & 0 & 0 & 96 & 4 & 0 \\
\midrule
angrybirds     & 90 & 0 & 10 & 22 & 76 & 2 \\
gameboy        & 98 & 0 & 2 & 6 & 94 & 0 \\
bullet         & 4 & 96 & 0 & 32 & 65 & 3 \\
lights         & 97 & 3 & 1 & 21 & 78 & 1 \\
FOTN           & 91 & 6 & 3 & 46 & 54 & 0 \\
monalisa       & 87 & 0 & 13 & 78 & 22 & 0 \\
\midrule
Overall        & 75.2 & 14.8 & 10.1 & 39.4 & 55.1 & 5.5 \\
\bottomrule
\end{tabular}
\nocaptionrule \caption{Indexed/Property Access Precision}
\label{fig:access_objects}
\end{figure}

\begin{figure}[ht]
\centering
\begin{tabular}{lrrrr}
\toprule
     & \multicolumn{2}{c}{Precision}
     & \multicolumn{2}{c}{Arithmetic} \\
\cmidrule(r){2-3}
\cmidrule{4-5}
Test & Poly (\%) & Ratio & Unknown (\%) & Ratio \\
\midrule
gmail          & 46 & 2.7 & 32 & 1.5 \\
googlemaps     & 38 & 3.2 & 23 & 4.6 \\
facebook       & 48 & 3.0 & 20 & 1.3 \\
flickr         & 61 & 6.1 & 39 & 4.9 \\
grooveshark    & 58 & 2.8 & 30 & 1.4 \\
meebo          & 36 & 3.6 & 28 & 3.5 \\
reddit         & 37 & 1.7 & 13 & 1.9 \\
youtube        & 40 & 6.7 & 28 & 7.0 \\
ztype          & 54 & 6.0 & 63 & 7.9 \\
280slides      & 76 & 4.0 & 93 & n/a \\
membench50     & 47 & 3.6 & 29 & 2.9 \\
\midrule
sunspider      & 5 & n/a & 6 & n/a \\
v8bench        & 18 & 2.6 & 1 & n/a \\
kraken         & 2 & n/a & 2 & n/a \\
\midrule
angrybirds     & 90 & n/a & 93 & n/a \\
gameboy        & 15 & 1.3 & 7 & 1.0 \\
bullet         & 62 & 3.9 & 79 & 11.3 \\
lights         & 37 & n/a & 63 & n/a \\
FOTN           & 28 & n/a & 57 & n/a \\
monalisa       & 44 & n/a & 41 & n/a \\
\midrule
Overall        & 42.1 & 4.3 & 37.4 & 6.0 \\
\bottomrule
\end{tabular}
\nocaptionrule \caption{Type Profiles Without Barriers}
\label{fig:without_barriers}
\end{figure}

\subsubsection{Access Site Precision}
\label{sec:access_objects}

Efficiently compiling indexed element and property accesses requires
knowledge of the kind of object being accessed.
This information is more specific than the monomorphic/polymorphic
distinction drawn in \Section\ref{sec:access_polymorphism}.
Figure~\ref{fig:access_objects} shows the fractions of indexed accesses on arrays
and of all property accesses which were optimized based on static knowledge.

In the indexed access section, the packed column shows the fraction
of operations known to be on packed arrays (\Section\ref{sec:supplemental-analyses}),
while the array column shows the fraction known to be on arrays not
known to be packed.
Indexed operations behave differently on arrays vs. other objects,
and avoiding dynamic array checks achieves some speedup.
The ``Uk'' column is the fraction of dynamic accesses on arrays which
are not statically known to be on arrays.

Static detection of array operations is very good on all kinds of sites,
with an average of 75.2\% of accesses on known packed arrays and
an additional 14.8\% on known but possibly not packed arrays.
A few outlier websites are responsible for the great majority of
accesses in the latter category.
For example, the V8 Crypto benchmark contains almost all of the benchmark's
array accesses, and the arrays used are not known to be packed
due to the top down order they are initialized.
Still, speed improvements on this benchmark are very large.

In the property access section of Figure~\ref{fig:access_objects},
the ``Def'' column shows the fraction
of operations which were statically resolved as definite properties
(\Section\ref{sec:supplemental-analyses}), while the PIC column shows the fraction which
were not resolved statically but were matched using a fallback mechanism,
polymorphic inline caches \cite{Holzle91}.
The ``Uk'' column is the fraction of operations which were not resolved
either statically or with a PIC and required a call into the VM;
this includes accesses where objects
with many different layouts are used, and accesses on rare kinds of
properties such as those with scripted getters or setters.

An average of 39.4\% of property accesses were resolved as definite properties,
with a much higher average proportion on benchmarks of 80.3\%.
The remainder were by and large handled by PICs, with only 5.5\% of
accesses requiring a VM call.
Together, these suggest that objects on websites are by and large
constructed in a consistent fashion, but that
our detection of definite properties needs
to be more robust on object construction patterns seen on websites but
not on benchmarks.

\subsubsection{Precision Without Barriers}
\label{sec:without_barriers}

To test the practical effect of using type barriers to improve precision,
we repeated the above website tests using a build of Firefox where
subset constraints were used in place of barrier constraints,
and type barriers were not used at all (semantic triggers were still used).
Some of the numbers from these runs are shown in Figure~\ref{fig:without_barriers}.

The precision section shows the fraction of indexed and property accesses which
were inferred as polymorphic, and the arithmetic section shows the fraction
of arithmetic operations where at least one operand type was unknown.
Both sections show the ratio of the given fraction to the comparable fraction
with type barriers enabled.
Overall, with type barriers disabled 42.1\% of accesses are polymorphic and 37.4\%
of arithmetic operations have operands of unknown type; precision is far worse
than with type barriers.
Benchmarks are affected much less than other kinds of sites;
these benchmarks use polymorphic structures much less
than the web at large.

%%% Local Variables: 
%%% mode: latex
%%% TeX-master: "main"
%%% End: 

\section{Related Work}
\label{sec:related-work}

There is an enormous literature on points-to analysis, JIT compilation, and
type inference. We only compare against a few here.

The most relevant work on type inference for JavaScript to the current work is
Logozzo and Venter's work on rapid atomic type analysis \cite{Logozzo10}.
Like ours, their analysis is also designed to be used online in the context of
JIT compilation and must be able to pay for itself. Unlike ours, their
analysis is purely static and much more sophisticated, utilizing a theory of
integers to better infer integral types vs floating point types. We eschew
sophistication in favor of simplicity and speed. Our evaluation shows that
even a much simpler static analysis, when coupled with dynamic checks, performs
very well ``in the wild''.
Our analysis is more
practical: we have improved handling of what Logozzo and Venter termed ``havoc''
statements, such as \code{eval}, which make static analysis results
imprecise. As Richards et al. argued in their surveys, real-world use of
\code{eval} is pervasive, between 50\% and 82\% for popular websites
\cite{Richards11, Richards10}.

Other works on type inference for JavaScript are more formal. The work of
Anderson et al. describes a structural object type system with subtyping over
an idealized subset of JavaScript \cite{Anderson05}. As the properties held
by JavaScript objects change dynamically,
the structural type of an object is a flow-sensitive property. Thiemann
and Jensen et al.'s typing frameworks approach this problem by using recency
types \cite{Thiemann05, Jensen09}. The work of Jensen et al. is in the context
of better tooling for JavaScript, and their experiments suggest that the
algorithm is not suitable for online use in a JIT compiler. Again, these
analyses do not perform well in the presence of statically uncomputable
builtin functions such as \code{eval}.

Performing static type inference on dynamic languages has been proposed at
least as early as Aiken and Murphy \cite{Aiken91}. More related in spirit to
the current work are the works of the the implementors of the Self language
\cite{Ungar87}. In implementing type inference for JavaScript, we faced many
challenges similar to what they faced decades earlier \cite{Ungar92,
  Agesen94}. Agesen outlines the design space for type inference algorithms
along the dimensions of efficiency and precision. We strived for an
algorithm that is both fast and efficient, at the expense of requiring
runtime checks when dealing with complex code.
Our experience building tracing JIT compilers \cite{GalVEE09, GalPLDI09} has
demonstrated that solely using type feedback limits the optimizations that we
can perform, and reaching peak performance requires static knowledge about
the possible types of heap values.

Agesen and H\"olzle compared the static approach of type inference with the
dynamic approach of type feedback and described the strengths and weaknesses
of both \cite{Agesen95}. Our system tries to achieve the best of both
worlds. The greatest difficulty in static type inference for polymorphic
dynamic languages, whether functional or object-oriented, is the need to
compute both data and control flow during type inference. We solve this by
using runtime information where static analyses do poorly,
e.g. determining the particular field of a polymorphic receiver or the
particular function bound to a variable. Our type barriers may be seen as a
type cast in context of Glew and
Palsberg's work on method inlining \cite{Glew02}.

Framing the type inference problem as a flow problem is a well-known approach
\cite{Oxhoj92, Palsberg91}; practical examples include Self's inferencer
\cite{Agesen93TI}.
%Generalizing Hindley/Milner-style type inference to type
%set inclusion from type set equality has been approached at least as early as
%Mitchell's work on type inference and Thatte's work on partial types
%\cite{Mitchell84, Thatte88}.
Aiken and Wimmers presented general results on
type inference using subset constraints \cite{Aiken93}.
%, and Wand showed how to
%frame inference of principal types in function languages as a constraint
%problem \cite{Wand87}.
%Our work, formally, is much simpler than many such
%proposed solutions to solving type constraints by choice.

% No
%unification is required to arrive a solution --- a solution always exists by
%simple propagation.

Other hybrid approaches to typing exist, such as Cartwright and Fagan's soft
typing and Taha and Siek's gradual typing \cite{Cartwright91, Siek07}. They
have been largely for the purposes of correctness and early error
detection, though soft typing has been successfully used to eliminate runtime checks \cite{Wright97}.
%Soft typing inserts run-time checks, and gradual typing focuses on
%the interaction of code with type annotation and code without
%annotation.
We say these approaches are at least partially \emph{prescriptive}, in that they
help enforce a typing discipline, while ours is entirely \emph{descriptive}, in that
we are inferring types only to help JIT compilation.


% Notes on other things to cite:
% We take efficiency of analyses to be used in JIT compilers as the ability to pay for themselves, reference chez scheme.
% Say somewhere and cite that pointer analysis scales well in the wild despite O(n^3).
% Future work: perhaps talk about pluggable static portions? parametrised. CFA2, Chambers's Fast Interprocedural Class Analysis paper

%%% Local Variables: 
%%% mode: latex
%%% TeX-master: "main"
%%% End: 

%\input{conclusion.tex}

\bibliography{citations}{}
\bibliographystyle{plain}

\end{document}
