\section{Analysis}

We present our analysis in two parts, the static ``may-have-type'' analysis
and the dynamic ``must-have-type'' analysis. The algorithm is based on
Andersen-style constraint-based pointer analysis. The static analysis, unlike
static type reconstruction algorithms, is intentionally \textit{incomplete}
with respect to the semantics of JavaScript. It does not account for all
possible behaviors of expressions and statements and only generates
constraints that model a ``may-have-type'' relation. All behaviors excluded by
the type constraints must be detected at runtime and their effects on types in
the program dynamically recorded. The analysis runs in the browser as functions
are trying to execute: code is analyzed function-at-a-time.

%% This paragraph and list doesn't seem to fit in the analysis section, this is more about performance. -ed
%
%Since the analysis runs in a browser as code is trying to execute, both time and space are at a premium and must be minimized. To be viable for deployment, our analysis must satisfy the following:
%
%\begin{itemize}
%
%\item Time and memory used must be linear in the size of the analyzed code.
%
%\item Querying results for an expression must take near-constant time, to avoid changing the performance characteristics of the compiler.
%
%% What does this mean? Interaction between "must" and "if only" is strange. -ed
%\item Results must be correct if only compiled code is analyzed.
%
%\end{itemize}

We describe constraint generation and checks required for a simplified core of
JavaScript expressions and statements, shown in Figure~\ref{fig:js-core}. We
let $f,x$ range over variables, $p$ range over property names, $i$ range over
integer literals, and $s$ range over string literals.

The types over which we are trying to infer is also shown in
Figure~\ref{fig:js-core}. The types can be primitive, \code{int},
\code{number}, \code{string}, an object type $o$, or a function
type\footnote{In full JavaScript, we also have the primitive types
  \code{bool}, \code{null}, and \code{undefined}.}. The \code{int} type
indicates a number expressible as a signed 32-bit integer, and is subsumed by
\code{number} --- \code{int} is added to all type sets containing
\code{number}. Finally, we have sets of types which the static analysis
computes. The domain and codomain of the function type are defined over type
sets.

\begin{figure}
\begin{align*}
v & ::= i\ |\ s\ |\ \text{\code{\{\}}} && \text{values}\\
e & ::= v\ |\ x\ |\ e + e\ |\ x.p\ |\ x[i]\ |\ f(x) && \text{expressions}\\
s & ::= x = e\ |\ x.p = e\ |\ x[i] = e && \text{statements}\\
\\
\tau & ::= \code{int}\ |\ \code{number}\ |\ \code{string}\ |\ o\ |\ T \rightarrow T && \text{types}\\
T & ::= \mathcal{P}(\tau) && \text{type sets}\\
\\
C & ::= T \supseteq T\ && \text{constraints}
\end{align*}
\caption{Simplified JavaScript Core, Types, and Constraints}
\label{fig:js-core}
\end{figure}

\begin{figure*}
\begin{align*}
\rulename{Int} & \inferrule{}{\judge^e i : T_i} &&
\left\{
\begin{array}{l}
T_i \supseteq \{\code{int}\}
\end{array}
\right\}\\
\rulename{String} & \inferrule{}{\judge^e s : T_s} &&
\left\{
\begin{array}{l}
T_s \supseteq \{\code{string}\}
\end{array}
\right\}\\
\rulename{Object} & \inferrule{}{\judge^e \text{\code{\{\}}} : T_{\text{\code{\{\}}}}} &&
\left\{
\begin{array}{l}
T_{\text{\code{\{\}}}} \supseteq \{o\}
\end{array}
\right\} \text{ where $o$ fresh}\\
\rulename{Var} & \inferrule{}{\judge^e x : T_x} && \emptyset\\
\rulename{Add} & \inferrule{\judge^e x : T_x ~~~ \judge^e y : T_y}{\judge^e x + y : T_{x+y}} &&
\left\{
\begin{array}{l}
T_{x+y} \supseteq \{\code{int}\}\ |\ \code{int} \in T_x \wedge \code{int} \in T_y,\\
T_{x+y} \supseteq \{\code{number}\}\ |\ \code{int} \in T_x \wedge \code{number} \in T_y,\\
T_{x+y} \supseteq \{\code{number}\}\ |\ \code{number} \in T_x \wedge \code{int} \in T_y,\\
T_{x+y} \supseteq \{\code{string}\}\ |\ \code{string} \in T_x \vee \code{string} \in T_y
\end{array}
\right\}\\
\rulename{Prop} & \inferrule{\judge^e x : T_x}{\judge^e x.p : T_{x.p}} &&
\left\{
\begin{array}{l}
T_{x.p} \supseteq \mathit{prop}(o,p)\ |\ o \in T_x
\end{array}
\right\}\\
\rulename{Index} & \inferrule{\judge^e x : T_x}{\judge^e x[i] : T_{x[i]}} &&
\left\{
\begin{array}{l}
T_{x[i]} \supseteq \mathit{index}(o)\ |\ o \in T_x
\end{array}
\right\}\\
\rulename{App} & \inferrule{\judge^e f : T_f ~~~ \judge^e x : T_x}{\judge^e f(x) : T_{f(x)}} &&
\left\{
\begin{array}{l}
T_{\mathit{dom}(f)} \supseteq T_x\ |\ T_{\mathit{dom}(f)} \rightarrow T_{\mathit{cod}(f)} \in T_f,\\
T_{f(x)} \supseteq T_{\mathit{cod}(f)}\ |\ T_{\mathit{dom}(f)} \rightarrow T_{\mathit{cod}(f)} \in T_f
\end{array}
\right\}\\
\rulename{AssignVar} & \inferrule{\judge^e x : T_x ~~~ \judge^e e : T_e}{\judge^s x = e : \text{\textbullet}} &&
\left\{
\begin{array}{l}
T_x \supseteq T_e
\end{array}
\right\}\\
\rulename{AssignProp} & \inferrule{\judge^e x : T_x ~~~ \judge^e e : T_e}{\judge^s x.p = e : \text{\textbullet}} &&
\left\{
\begin{array}{l}
\mathit{prop}(o,p) \supseteq T_e\ |\ o \in T_x
\end{array}
\right\}\\
\rulename{AssignIndex} & \inferrule{\judge^e x : T_x ~~~ \judge^e e : T_e}{\judge^s x[i] = e : \text{\textbullet}} &&
\left\{
\begin{array}{l}
\mathit{index}(o) \supseteq T_e\ |\ o \in T_x
\end{array}
\right\}
\end{align*}
\caption{Constraint Generation Rules}
\label{fig:constraint-rules}
\end{figure*}

Section~\ref{sec:object-types} describes object types in more detail,
Section~\ref{sec:constraints} describes generation of the type constraints
which forms the static portion of the analysis, and Section~\ref{XXX}
describes type barriers and the other runtime checks which form the dynamic
portion of the analysis.

\subsection{Object Types}
\label{sec:object-types}

To reason about the effects of property accesses, we need type information
for JavaScript objects and their properties.
We associate every JavaScript object with an {\it object type} $o$.
When $o \in T_e$ for some expression $e$, then the possible values
for $e$ when it is executed include all JavaScript objects with type $o$.

Types are assigned to JavaScript objects according to their prototype:
all JavaScript objects with the same type have the same prototype.
Additionally, objects with the same prototype have the same type,
except for plain \code{Object}, \code{Array} and \code{Function} objects.
\code{Object} and \code{Array} objects have the same type if they were
allocated at the same location in a script,
and \code{Function} objects have the same type if they are closures
for the same script.

For the sake of brevity and ease of exposition, our simplified JavaScript core
only contains the ability to construct \code{Object}-prototyped object
literals via the \code{\{\}} syntax.

The type of an object is nominal: it is independent from the properties it
has. Objects which are structurally identical may have different types, and
objects with the same type may have different structures. This is crucial for
efficient analysis. Properties can be added or deleted to JavaScript objects
at any time, and using structural typing would make an object's type a
flow-sensitive property.

Instead, for each object type we compute the {\it possible} properties which
objects of that type can have, and the possible types of those properties.
These are represented as type sets $\mathit{prop}(o,p)$ and
$\mathit{index}(o)$. The set $\mathit{prop}(o,f)$ captures the possible types
of a non-integer property $p$ for objects with type $o$, while
$\mathit{index}(o)$ captures the possible types of all integer properties of
all objects with type $o$.

\subsection{Type Constraints}
\label{sec:constraints}

The static portion of our analysis generates constraints on the flow of types
through the program. We model the semantics of JavaScript by assigning to each
expression a \emph{type set} which denotes the set of types that it may
have. It is important to note that these statically-generated constraints do
not model the entirety of JavaScript semantics. Not every type that the
expression may have is in the statically determined set.

The syntax of constraints are shown in Figure~\ref{fig:js-core}. Our only
constraint is the subset constraint. Rules for the constraint generation
functions, $\mathcal{C}_e(e)$ for expressions and $\mathcal{C}_s(s)$ for
statements, are shown in Figure~\ref{fig:constraint-rules}. Statically
analyzing a method, then, is taking the union of the results from applying
$\mathcal{C}_s$ to every statement in the method.

The \textsc{Int}, \textsc{String}, and \textsc{Object} rules for literals and
the \textsc{Var} rule for variables are straightforward.

The \textsc{Add} rule is complex, as addition in JavaScript is very complicated. It is
defined for any combination of values, can perform either a numeric addition,
string concatenation, or even function calls if either of its operands is an
object (calling their \code{valueOf} or \code{toString} members, producing a
number or string).

Using incomplete modeling lets us cut through this complexity. The vast
majority of dynamic additions in JavaScript programs (NUMBERS) are adding two
numbers or concatenating two strings. We statically model exactly these cases,
and use runtime checks to monitor the results produced by adding other
combinations of values, at little runtime cost.

Note that even the integer addition rule we have given is incomplete: the
result will be marked as an integer, ignoring the possibility of
overflow. When addition of two integers overflows, the result is not
expressible in 32 bits, and has type \code{number}. Overflow is extremely
rare, and accounting for it statically would require us to mark the result of
all additions as arbitrary numbers, except in the limited cases where we can
prove no overflow is possible.

Ignoring overflow statically forces us to use runtime checks when overflows
occur. There is no extra runtime cost associated with these checks, as the
code generated for integer additions requires an overflow check regardless.

TODO explanation for other rules.

The reader may wonder where function types come from, as our simplified
JavaScript core does not contain function declarations. Recall that the
analysis is run function-at-a-time. At the end of analysis, the type sets of
the formal parameters and the return value are gathered and a function type is
constructed for the current function. We omit this detail from the
formalization.

\subsection{Type Barriers}

TODO

As described in Section~\ref{XXX}, type barriers are a form of runtime
check designed to retain precision when a program contains polymorphic code.

TODO

For any subset constraint $A \subseteq B$, we are free to restrict
propagation of types which are in $A$ but have never been observed for $B$,
so long as we include a runtime check for values read into $B$ and update
it with any newly observed types.

The presence or absence of type barriers for a given subset constraint
is not monotonic with respect to the contents of the type sets in the program.
As new types are discovered, new type barriers may be required, but existing
ones may become unnecessary.
However, it is always safe to perform the runtime check for a given barrier,
albeit less efficient.

\subsection{Interpreter Warmup}

TODO

\subsection{Supplemental Analysis}

TODO

%%% Local Variables: 
%%% mode: latex
%%% TeX-master: "main"
%%% End: 
